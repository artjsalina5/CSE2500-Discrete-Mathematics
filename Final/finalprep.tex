% copyright arturo salinas-aguayo 2025
\documentclass[12pt]{article}

\usepackage{graphicx}
\usepackage{amsmath}
\usepackage{amssymb}
\usepackage{amsthm}
\usepackage{array}
\usepackage{mdframed}
\usepackage{amsfonts}
\usepackage{fancyhdr}
\usepackage{geometry}
\usepackage{subfigure}
\usepackage{caption}
\usepackage{tikz}
\usepackage{bm}
\usepackage{float}

\geometry{letterpaper, margin=1in}
\graphicspath{ {../images/} }

% Header and Footer
\pagestyle{fancy}
\fancyhf{}
\fancyhead[L]{CSE 2500-01: Homework 9}
\fancyhead[R]{Page \thepage}
\setlength{\headheight}{15pt}

\author{Arturo Salinas-Aguayo}
\title{CSE 2500-01: Homework 9}

\usepackage{etoolbox} % Required for \ifstrempty

\newcounter{theo}[section]
\newenvironment{theo}[1][]{%
  \stepcounter{theo}%
  \ifstrempty{#1}%
    {\mdfsetup{%
      frametitle={%
        \tikz[baseline=(current bounding box.east), outer sep=0pt]
          \node[anchor=east,rectangle,fill=blue!20]
          {\strut Theorem~\thetheo};}}%
    }%
    {\mdfsetup{%
      frametitle={%
        \tikz[baseline=(current bounding box.east), outer sep=0pt]
          \node[anchor=east,rectangle,fill=blue!20]
          {\strut Theorem~\thetheo:~#1};}}%
    }%
  \mdfsetup{
    innertopmargin=10pt,
    linecolor=blue!20,
    linewidth=2pt,
    topline=true,
    frametitleaboveskip=-\ht\strutbox,
  }
  \begin{mdframed}[]\relax%
}{\end{mdframed}}

\renewcommand\qedsymbol{\textbf{QED}}
\newtheorem{example}{Example}
% Example block environment
\newenvironment{examp}
{\vspace{0.5cm}
 \hrule
\vspace{0.5cm}
\begin{example}}
{\hrule
\vspace{0.5cm}
\end{example}}
% Macro for centered equals in final row
\newcommand{\centereq}[2]{#1 \kern-3em = \hphantom{#1} #2}

% Macro for vertical LHS/RHS derivation layout
\newenvironment{lhsrhsproof}[0]{
  \[
  \renewcommand{\arraystretch}{1.5}
  \begin{array}{c@{\quad\quad}c}
  \textbf{LHS} & \textbf{RHS} \\
}{
  \end{array}
  \]
}
\begin{document}
\newcommand{\closure}[2][3]{%
	{}\mkern#1mu\overline{\mkern-#1mu#2}}
\newcommand\ncoverline[1]{\mkern1mu\overline{\mkern-1mu#1\mkern-1mu}\mkern1mu}
% Title Page
\begin{titlepage}
	\centering
	\vspace*{3cm}
	\huge\textbf{CSE 2500-01: Homework 9}\\
	\vspace{5cm}
	\Large\textbf{Arturo Salinas-Aguayo}\\
	\normalsize
	Spring 2025\\
	Electrical and Computer Engineering Department\\
	\vfill
	\includegraphics[scale=0.1]{uconnlogo}\\
	College of Engineering, University of Connecticut\\
	\scriptsize{Coded in \LaTeX}
	\vspace*{1cm}
\end{titlepage}
\section{Previous Test Questions}
\begin{enumerate}
	\item Assume $p \rightarrow q$ is False. Find truth value of:
	      \[
		      (\sim{}p \land q) \leftrightarrow (p \lor q)
	      \]
	      \begin{theo}
		      If $p\leftarrow q$ is false, then this means that $(\sim p \lor q)$ is false.\\
		      The negation of $(\sim p \lor q)$ is $(p \land \sim q)$. which is true with the supposition.\\
		      Therefore, since $p$ is true, $q$ must be false.\\
		      The biconditional states that both sides must be true or both sides must be false.\\
		      The left side is true, and the right side is false.\\
		      Therefore, the biconditional is false.
	      \end{theo}
	\item $\forall n \in \mathbb{Z}, n \geq 1, a^n - b^n$ is a multiple of $a - b$
	      \begin{theo}
		      \begin{proof}[Proof by Mathematical Induction]
			      Let $P(n)$ be the property:
			      \[
				      a - b \mid a^n - b^n
			      \]
			      \textbf{Basis Step}: Check if $P(1)$ is true:\\
			      \[a^0 - b^0 = a - b\]
			      This is true.\\
			      \textbf{Inductive Step}: Assume $P(k)$ is true for some $k$ with $k \geq 1$.
			      That is,
			      \[
				      a^k - b^k = (a - b) \cdot m \text{ for some } m \in \mathbb{Z}
			      \]
			      It must be shown that $P(k + 1)$ is also true.
			      \begin{align*}
				      a^{k+1} - b^{k+1} & = a^{k+1} - ab^{k} + ab^{k} - b^{k+1} & \text{by adding and subtracting $ab^k$} \\
				                        & = a(a^k - b^k) + b^k(a - b)           & \text{by factoring}                     \\
				                        & = a(a-b)m + b^k(a-b)                  & \text{by inductive hypothesis}          \\
				                        & = (a-b)[a(m+b^k)]
			      \end{align*}
			      Now let $t = a(m + b^k)$, which is an integer due to subtraction of integers.\\
			      Therefore, $a^{k+1} - b^{k+1} = (a - b) \cdot t$\\
		      \end{proof}
	      \end{theo}
	\item $n^3 - n$ is divisible by 6, for each integer $n \geq 0$
	      \begin{theo}
		      \begin{proof}[Proof by Mathematical Induction]
			      Let $P(n)$ by the expression:
			      \[
				      6 \mid n^3 - n
			      \]
			      \textbf{Basis Step}: $P(0)$ is:
			      \[
				      0^3 - 0 = 0
			      \]
			      $0$ is divisible by $6$.\\
			      \textbf{Inductive Step}: Assume $P(k)$ is true, that is:
			      \[
				      k^3 - k = 6\cdot m \quad\text{for some integer $m$}
			      \]
			      We must show $P(k+1)$ is true, that is,
			      \[
				      (k+1)^3 - (k+1) = 6 \cdot m
			      \]
			      \begin{align*}
				      (k+1)^3 - (k+1) & = k^3 + 3k^2 + 3k + 1 - k - 1 \\
				                      & = (k^3-k) + 3(k^2 + k)        \\
				                      & = 6 \cdot (m) + 3(k(k+1))     \\
			      \end{align*}
			      Since $k(k+1)$ is even then let it equal $2s$ for some integer $s$.
			      Then:
			      \begin{align*}
				      (k+1)^3 - (k+1) & = 6m + 3 \cdot 2s \\
				                      & = 6(m + s)        \\
			      \end{align*}
			      Thus $(k+1)^3 - (k+1)$ is divisible by 6.
		      \end{proof}
	      \end{theo}
\end{enumerate}

\end{document}
% vim: set ft=tex tw=80 ts=2 sts=2 sw=2 noet spell:
