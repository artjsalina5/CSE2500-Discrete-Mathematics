% copyright arturo salinas-aguayo 2024
\documentclass[12pt]{article}

\usepackage{graphicx}
\usepackage{amsmath}
\usepackage{array}
\usepackage{amsfonts}
\usepackage{fancyhdr}
\usepackage{geometry}
\usepackage{subfigure}
\usepackage{caption}
\usepackage{tikz}
\usepackage{bm}
\usepackage{float}

\geometry{letterpaper, margin=1in}
\graphicspath{ {../images/} }

% Header and Footer
\pagestyle{fancy}
\fancyhf{}
\fancyhead[L]{CSE 2500-01: Homework 2}
\fancyhead[R]{Page \thepage}
\setlength{\headheight}{15pt}

\author{Arturo Salinas-Aguayo}
\title{CSE 2500-01: Homework 2}
% theorem set
\newtheorem{example}{Example}
% Example block environment
\newenvironment{examp}
{\vspace{0.5cm}
 \hrule
\vspace{0.5cm}
\begin{example}}
{\hrule
\vspace{0.5cm}
\end{example}}

\begin{document}
\newcommand{\closure}[2][3]{%
	{}\mkern#1mu\overline{\mkern-#1mu#2}}
\newcommand\ncoverline[1]{\mkern1mu\overline{\mkern-1mu#1\mkern-1mu}\mkern1mu}
% Title Page
\begin{titlepage}
	\centering
	\vspace*{3cm}
	\huge\textbf{CSE 2500-01: Homework 2}\\
	\vspace{5cm}
	\Large\textbf{Arturo Salinas-Aguayo}\\
	\normalsize
	Spring 2025\\
	Electrical and Computer Engineering Department\\
	\vfill
	\includegraphics[scale=0.1]{uconnlogo}\\
	College of Engineering, University of Connecticut\\
	\scriptsize{Coded in \LaTeX}
	\vspace*{1cm}
\end{titlepage}
\begin{enumerate}
	\item Prove that:
	      \[ (p \land q) \lor (\sim p \land q) \lor (\sim p \land \sim q)
	      \]
	      Using logical equivalences derived from Theorem 2.1.1. 
	      
	      Indicate the used Logical Equivalence in each step. Do not use a truth table.
	      
\end{enumerate}
\end{document}
% vim: set ft=tex tw=80 ts=2 sts=2 sw=2 noet:
