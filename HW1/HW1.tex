% copyright arturo salinas-aguayo 2024
\documentclass[12pt]{article}

\usepackage{graphicx}
\usepackage{amsmath}
\usepackage{array}
\usepackage{amsfonts}
\usepackage{fancyhdr}
\usepackage{geometry}
\usepackage{subfigure}
\usepackage{caption}
\usepackage{tikz}
\usepackage{bm}
\usepackage{float}

\geometry{letterpaper, margin=1in}
\graphicspath{ {../images/} }

% Header and Footer
\pagestyle{fancy}
\fancyhf{}
\fancyhead[L]{CSE 2500-01: Homework 1}
\fancyhead[R]{Page \thepage\ of 2}
\setlength{\headheight}{15pt}

\author{Arturo Salinas-Aguayo}
\title{CSE 2500-01: Homework 1}
% theorem set
\newtheorem{example}{Example}
% Example block environment
\newenvironment{examp}
{\vspace{0.5cm}
 \hrule
\vspace{0.5cm}
\begin{example}}
{\hrule
\vspace{0.5cm}
\end{example}}

\begin{document}
\newcommand{\closure}[2][3]{%
	{}\mkern#1mu\overline{\mkern-#1mu#2}}
\newcommand\ncoverline[1]{\mkern1mu\overline{\mkern-1mu#1\mkern-1mu}\mkern1mu}
% Title Page
\begin{titlepage}
	\centering
	\vspace*{3cm}
	\huge\textbf{CSE 2500-01: Homework 1}\\
	\vspace{5cm}
	\Large\textbf{Arturo Salinas-Aguayo}\\
	\normalsize
	Spring 2025\\
	Electrical and Computer Engineering Department\\
	\vfill
	\includegraphics[scale=0.1]{uconnlogo}\\
	College of Engineering, University of Connecticut\\
	\scriptsize{Coded in \LaTeX}
	\vspace*{1cm}
\end{titlepage}

\section*{}
\begin{enumerate}
	\item (10 points) Restate the statement ``Every positive number has a positive square root'' by filling in the blank:
	      \begin{enumerate}
		      \item All positive numbers have have a positive square root.
		      \item For a positive number \( e \), there is a positive
		            square root for \( e \).
		      \item For all positive numbers \( e \), there is a positive
		            number \( r \) such that its square is positive..
	      \end{enumerate}
	\item (20 points)
	      \begin{enumerate}
		      \item Is \( 2 \in \{2\} \)?
		            
		            Yes, 2 is an element in the set.
		      \item How many elements are in the set \( \{2, 2, 2, 2\} \)?
		            
		            There is one element in the set.
		      \item How many elements are in the set \( \{0, \{0\}\} \)?
		            
		            There are two elements in this set.
		      \item Is \( \{0\} \in \{\{0\}, \{1\}\} \)?
		            
		            Yes the set of 0 is in the set.
		      \item Is \( 0 \in \{\{0\}, \{1\}\} \)?
		            
		            No, the element 0 is not in the set.
	      \end{enumerate}
	\item (15 points) Let \( S = \{1, 3, 10, 20\} \) and \( T = \{1, 10\} \).
	      \begin{enumerate}
		      \item Is \( \emptyset \subseteq S \)?
		            
		            Yes, the empty set is in the set S.
		      \item Is \( \emptyset \subseteq T \)?
		            
		            Yes, the empty set in in the set T.
		      \item Is \( S \subseteq T \)?
		            
		            No, the set S is not a subset of set T.
		      \item Is \( T \subseteq S \)?
		            
		            Yes, the set T is a subset of S.
		      \item Is \( T \) a proper subset of \( S \)? How is this question different from part d)?
		            
		            A Proper Subset is a term which defines the case in
		            which every element of the compared set is in the
		            target set, however, there is at least one element
		            that is not in the compared set that is in the target
		            set.
		            
		            Yes, the set T is a proper subset of S.
	      \end{enumerate}
	\item (10 points) Let \( S = \{2, 4, 6\} \) and \( T = \{1, 3, 5\} \). Use the set-roster notation to write the following sets:
	      \begin{enumerate}
		      \item \( S \times T \)
		            
		            \[
			            S \times T = \{(2, 1), (2, 3), (2, 5), (4, 1), (4, 3), (4, 5), (6, 1), (6, 3), (6, 5)\}
		            \]
		            
		      \item \( S \times S \)
		            
		            \[
			            S \times S = \{(2, 2), (2, 4), (2, 6), (4, 2), (4, 4), (4, 6), (6, 2), (6, 4), (6, 6)\}
		            \]
	      \end{enumerate}
	\item (25 points) Let \( G = \{-2, 0, 2\} \) and \( H = \{4, 6, 8\} \), and define a relation \( V \) from \( G \) to \( H \) as follows: for all \( (x, y) \in G \times H \), \( (x, y) \in V \) if \( \frac{x-y}{4} \) is an integer. Answer the following:
	      \begin{enumerate}
		      \item Is \( 2 \, V \, 6 \)?
		            
		            This is equivalent to \(\frac{2-6}{4}\) and checks that
		            the outcome is an integer. 
		            
		            \(\frac{2-6}{4} = -1\) , \(-1\) is an integer,
		            therefore this is 2 is related to 6 by V.
		      \item Is \( (-2) \, V \, (6) \)?
		            
		            This is asking if \(\frac{-2-6}{4}\) outputs an
		            integer.
		            
		            The outcome is \(\frac{-2-6}{4} = -2\) which is
		            an integer, therefore -2 is related to 6 by V.
		      \item Is \( (0, 6) \in V \)?
		            
		            This is asking if 0 and 6 are related by V.
		            
		            The outcome is \(\frac{0-6}{4} = \frac{-3}{2}\)
		            this is not an integer and therefore 0 and 6 are
		            not related by V
		            
		      \item Is \( (2, 4) \in V \)?
		            
		            This is asking if 2 and 4 are related by 4.
		            
		            The outcome is \(\frac{2-4}{4} = \frac{-1}{2}\) which is
		            not an integer, therefore 2 and 4 are not related by V.
		      \item Write \( V \) as a set of ordered pairs.
		            
		            
		            \vspace{5cm}
		            
		            
		            
		      \item Write the domain and co-domain of \( V \).
		            
		            \vspace{5cm}
		            
		            
		      \item Draw an arrow diagram for \( V \).
		            \vspace{5cm}
		            
		            
		            
	      \end{enumerate}
	\item (10 points) Let \( X = \{2, 4, 5\} \) and \( Y = \{1, 2, 4, 6\} \). For each of the following relations, draw an arrow diagram and say whether the relation is a function:
	      \begin{enumerate}
		      \item \( R = \{(2, 6), (4, 2), (5, 2)\} \)
		            \vspace{5cm}
		      \item \( V = \{(2, 4), (4, 1), (4, 2), (5, 6)\} \)
		            \vspace{5cm}
	      \end{enumerate}
	\item (10 points) Define functions \( H \) and \( K \) from \( \mathbb{R} \) to \( \mathbb{R} \) by the following formulae: for all \( x \in \mathbb{R} \),
	      \[
		      H(x) = (x - 2)^2
	      \]
	      \[
		      K(x) = (x - 1)(x - 3) + 1
	      \]
	      \begin{enumerate}
		      \item Are these the same function? Why or why not?
	      \end{enumerate}
\end{enumerate}

\end{document}
% vim: set ft=tex tw=80 ts=2 sts=2 sw=2 noet:
