% copyright arturo salinas-aguayo 2025
\documentclass[12pt]{article}

\usepackage{graphicx}
\usepackage{amsmath}
\usepackage{amssymb}
\usepackage{amsthm}
\usepackage{array}
\usepackage{mdframed}
\usepackage{amsfonts}
\usepackage{fancyhdr}
\usepackage{geometry}
\usepackage{subfigure}
\usepackage{caption}
\usepackage{tikz}
\usepackage{bm}
\usepackage{float}

\geometry{letterpaper, margin=1in}
\graphicspath{ {../images/} }

% Header and Footer
\pagestyle{fancy}
\fancyhf{}
\fancyhead[L]{CSE 2500-01: Homework 9}
\fancyhead[R]{Page \thepage}
\setlength{\headheight}{15pt}

\author{Arturo Salinas-Aguayo}
\title{CSE 2500-01: Homework 9}

\usepackage{etoolbox} % Required for \ifstrempty

\newcounter{theo}[section]
\newenvironment{theo}[1][]{%
  \stepcounter{theo}%
  \ifstrempty{#1}%
    {\mdfsetup{%
      frametitle={%
        \tikz[baseline=(current bounding box.east), outer sep=0pt]
          \node[anchor=east,rectangle,fill=blue!20]
          {\strut Theorem~\thetheo};}}%
    }%
    {\mdfsetup{%
      frametitle={%
        \tikz[baseline=(current bounding box.east), outer sep=0pt]
          \node[anchor=east,rectangle,fill=blue!20]
          {\strut Theorem~\thetheo:~#1};}}%
    }%
  \mdfsetup{
    innertopmargin=10pt,
    linecolor=blue!20,
    linewidth=2pt,
    topline=true,
    frametitleaboveskip=-\ht\strutbox,
  }
  \begin{mdframed}[]\relax%
}{\end{mdframed}}

\renewcommand\qedsymbol{\textbf{QED}}
\newtheorem{example}{Example}
% Example block environment
\newenvironment{examp}
{\vspace{0.5cm}
 \hrule
\vspace{0.5cm}
\begin{example}}
{\hrule
\vspace{0.5cm}
\end{example}}
% Macro for centered equals in final row
\newcommand{\centereq}[2]{#1 \kern-3em = \hphantom{#1} #2}

% Macro for vertical LHS/RHS derivation layout
\newenvironment{lhsrhsproof}[0]{
  \[
  \renewcommand{\arraystretch}{1.5}
  \begin{array}{c@{\quad\quad}c}
  \textbf{LHS} & \textbf{RHS} \\
}{
  \end{array}
  \]
}
\begin{document}
\newcommand{\closure}[2][3]{%
	{}\mkern#1mu\overline{\mkern-#1mu#2}}
\newcommand\ncoverline[1]{\mkern1mu\overline{\mkern-1mu#1\mkern-1mu}\mkern1mu}
% Title Page
\begin{titlepage}
	\centering
	\vspace*{3cm}
	\huge\textbf{CSE 2500-01: Homework 9}\\
	\vspace{5cm}
	\Large\textbf{Arturo Salinas-Aguayo}\\
	\normalsize
	Spring 2025\\
	Electrical and Computer Engineering Department\\
	\vfill
	\includegraphics[scale=0.1]{uconnlogo}\\
	College of Engineering, University of Connecticut\\
	\scriptsize{Coded in \LaTeX}
	\vspace*{1cm}
\end{titlepage}

\section*{Problems}
\begin{enumerate}
	\item Let \( A = \{n \in \mathbb{Z} \mid n = 5r, \exists r \in \mathbb{Z}\} \) and \( B = \{m \in \mathbb{Z} \mid m = 20s, \exists s \in \mathbb{Z}\} \).
	      \begin{theo}[Proofs of Set Inclusion]
		      \begin{enumerate}
			      \item \( A \subseteq B \)\\
			            No, \(A \not \subseteq B\).
			            \begin{proof}[Disproof by Counterexample]
				            \(A \not \subseteq B\). For example $5 \in A$ because $5 = 5\cdot1$, but $5 \neq 20k$ for any integer $k$. Therefore, $5 \notin B$.
			            \end{proof}
			      \item \( B \subseteq A \)
			            Yes, every element of $B$ is in $A$.\\
			            \begin{proof}
				            Suppose $n$ is any element of $B$.\\
				            $n = 20k$ for some integer $k$, and $n = 5(4k)$. \\
				            So, since $4k$ is an integer, $n \in A$.
			            \end{proof}
		      \end{enumerate}
	      \end{theo}
	\item Let \( A = \{a, b, c\} \), \( B = \{b, c, d\} \), and \( C = \{b, c, e\} \)
	      \begin{theo}[Union and Intersect Operations]
		      \begin{enumerate}
			      \item Find \( A \cap (B \cup C) \), \( (A \cap B) \cup C \), and \( (A \cap B) \cup (A \cap C) \). Which of these sets are equal?
			            \begin{enumerate}
				            \item \( A \cap (B \cup C) = \{a,b,c\} \cap \{b,c,d,e\} = \{b,c\}\)
				            \item \( (A \cap B) \cup C = \{b,c\} \cup \{b,c,e\} = \{b,c,e\} \)
				            \item \( (A \cap B) \cup (A \cap C) = \{b,c\} \cup \{b, c\} = \{b,c\} \)
				            \item Hence, \( A \cap (B \cup C) = (A \cap B) \cup(A\cap C)\)
			            \end{enumerate}
			      \item Find \( (A - B) - C \) and \( A - (B - C) \). Are these two sets equal?
			            \begin{enumerate}
				            \item \( (A - B) - C = \{a\} - \{b,c,e\} = \{a\} \)
				            \item \( A - (B - C) = \{a,b,c\} - \{d\} = \{a,b,c\}\)
				            \item No, these sets are not equal.
			            \end{enumerate}
		      \end{enumerate}
	      \end{theo}
	      \newpage
	\item Let \( S_i = \{x \in \mathbb{R} \mid 1 < x < 1 + \frac{1}{i}\} = (1, 1+\frac{1}{i}) \) for all positive integers \( i \).
	      \begin{theo}[Set Operations]
		      \begin{enumerate}
			      \item \( \bigcup\limits_{i=1}^{\infty} S_i = \bigcup\limits_{i=1}^{\infty} (1, 1 + \frac{1}{i}) = (1, 2) = S_1\)
			      \item \( \bigcap\limits_{i=1}^{\infty} S_i = \bigcup\limits_{i=1}^{\infty} (1,1 + \frac{1}{i}) = \emptyset\)
		      \end{enumerate}
	      \end{theo}

	\item Identify if the following sets are partitions:
	      \begin{theo}[Partitions]
		      \begin{enumerate}
			      \item Is \(\{\{5,4\}, \{7,2\}, \{1,3,4\}, \{6,8\}\}\) a partition of \( A = \{1,2,3,4,5,6,7,8\} \)?\\
			            No, because \(4\) is in two sets, \(\{5,4\}\) and \(\{1,3,4\}\).
			      \item Is \(\{\{1,5\}, \{4,7\}, \{2,8,6,3\}\}\) a partition of \( A = \{1,2,3,4,5,6,7,8\} \)?\\
			            Yes, every element in \{1,2,3,4,5,6,7,8\} is in exactly one set and no element is in more than one set of the partition.
		      \end{enumerate}
	      \end{theo}

	\item Find \( \mathcal{P}(\mathcal{P}(\mathcal{P}(\emptyset))) \).\\
	      \begin{theo}[Power Sets]
		      \[\mathcal{P}(\mathcal{P}(\mathcal{P}(\emptyset))) = \{\emptyset, \{\emptyset\}, \{\{\emptyset\}\}, \{\emptyset, \{\emptyset\}\}\}\]
	      \end{theo}

	\item Let \( A_1 = \{1,2,3\} \), \( A_2 = \{u, v\} \), and \( A_3 = \{m, n\} \). Find the following sets:
	      \begin{theo}[Set Products]
		      \begin{enumerate}
			      \item \((A_1 \times A_2) \times A_3\)\\
			            \begin{align*}
				            (A_1 \times A_2) \times A_3 = \{ & ((1,u),m),\ ((1,u),n),\ ((1,v),m),\ ((1,v),n),   \\
				                                             & ((2,u),m),\ ((2,u),n),\ ((2,v),m),\ ((2,v),n),   \\
				                                             & ((3,u),m),\ ((3,u),n),\ ((3,v),m),\ ((3,v),n) \}
			            \end{align*}
			      \item \(A_1 \times A_2 \times A_3\)\\
			            \begin{align*}
				            A_1 \times A_2 \times A_3 = \{ & (1,u,m),\ (1,u,n),\ (1,v,m),\ (1,v,n),   \\
				                                           & (2,u,m),\ (2,u,n),\ (2,v,m),\ (2,v,n),   \\
				                                           & (3,u,m),\ (3,u,n),\ (3,v,m),\ (3,v,n) \}
			            \end{align*}
			      \item How are these sets different?
			            \begin{itemize}
				            \item In part (a), each element is a pair, where the first component is itself a pair. The structure is \(((x,y),z)\).j
				            \item In part (b), each element is a triple \((x,y,z)\).
			            \end{itemize}
		      \end{enumerate}
	      \end{theo}
	\item Use an elementary argument to prove the following statement using contrapositive. You may assume that all subsets are sets of a universal set \( U \). It may be helpful to proceed by division into cases. For all sets \( A, B, C \), if \( A \subseteq C \) and \( B \subseteq C \), then \( (A \cup B) \subseteq C \).
	      \begin{theo}[Contrapositive Proof]
		      For all sets \( A, B, C \), if \( A \subseteq C \) and \( B \subseteq C \), then \( (A \cup B) \subseteq C \).
		      \begin{proof}[Proof by Contrapositive]
			      Suppose $x \in A \cup B$. \\
			      By definition of union, $x \in A$ or $x
				      \in B$.\\
			      \begin{itemize}
				      \item \textbf{Case 1:} \( x \in A \).
				            $x \in C$ because $A \subseteq C$.
				      \item \textbf{Case 2:} \( x \in B \).
				            $x \in C$ because $B \subseteq C$.
			      \end{itemize}
			      Therefore, by contrapositive:\\
			      If \( A \subseteq C \) and \( B \subseteq C \), then \( (A \cup B) \subseteq C \).
		      \end{proof}
	      \end{theo}
	      \newpage
	\item Use the element method for proving a set equals the empty set to prove: For all sets \( A, B, C \), if \( A \subseteq B \) and \( B \cap C = \emptyset \), then \( A \cap C = \emptyset \).
	      \begin{theo}[Proving a Set Equals $\emptyset$]
		      For all sets \( A, B, C \), if \( A \subseteq B \) and \( B \cap C = \emptyset \), then \( A \cap C = \emptyset \).
		      \begin{proof}[Proof by Contradiction]
			      Let \( A, B, C \) be sets such that \( A \subseteq B \) and \( B \cap C = \emptyset \).\\
			      Suppose \( A \cap C \neq \emptyset \). Then there is an element \( x \) such that \( x \in A \cap C \). \\
			      By definition of intersection, \\
			      \[ x \in A \quad\text{and}\quad  x \in C \]
			      Since \( A \subseteq B \), it follows that \( x \in B \). However, since \( B \cap C = \emptyset \), we have \( x \notin C \). \\
			      This is a contradiction.\\
			      Therefore, our assumption that \( A \cap C \neq \emptyset \) must be false. Thus, we conclude that \( A \cap C = \emptyset \).
		      \end{proof}
	      \end{theo}

	\item Use the element method for proving a set equals the empty set to prove: For all sets \( A, B, C \), if \( B \cap C \subseteq A \), then \( (C - A) \cap (B - A) = \emptyset \).
	      \begin{theo}[Proving a Set Equals $\emptyset$]
		      For all sets \( A, B, C \), if \( B \cap C \subseteq A \), then \( (C - A) \cap (B - A) = \emptyset \).
		      \begin{proof}[Proof by Contradiction]
			      Let $A, B,$ and $C$ be any sets such that \( B \cap C \subseteq A\).\\
			      Suppose \((C - A) \cap (B - A) \neq \emptyset\).\\
			      By definition of empty set, there is an element $x$ such that \[ x \in (C - A) \cap (B - A) = \emptyset\]
			      By definition of intersection, \[x \in C - A \quad\text{and}\quad x \in B - A\]
			      By definition of set difference, \(x \in C\) and \(x \not \in A\) and \(x \in B\) and \(x \not \in A\).\\
			      Since \( x \in C\) and \(x \in B\), \\
			      \[x \in B \cap C\] by the definition of intersection.\\
			      But \(B \cap C \subseteq A\), so \(x \in A\).\\
			      This is a contradiction, since we have shown that \(x \in A\) and \(x \not \in A\).\\
			      Hence the supposition is false, and we conclude that \((C - A) \cap (B - A) = \emptyset\).
		      \end{proof}
	      \end{theo}

	      \newpage
	\item Construct an algebraic proof for the following (cite a set identity to justify each step): For all sets \( A, B, C \),
	      \[
		      (A - B) - (B - C) = A - B
	      \]
	      \begin{theo}[Algebraic Proof]
		      \begin{proof}[Proof by Algebraic Properties]
			      Let $A$, $B$, and $C$ be any sets.\\
			      \((A - B) - (B - C)\)
			      \begin{align*}
				       & = (A \cap B^c) \cap (B \cap C^c)^c                   & \text{by definition of set difference} \\
				       & = (A \cap B^c) \cap (B^c \cup C)                     & \text{by De Morgan's Law}              \\
				       & = (A \cap B^c) \cap (B^c \cap C)                     & \text{by double complement
				      law}                                                                                             \\
				       & = ((A \cap B^c) \cap B^c) \cap ((A \cap B^c) \cap C) & \text{by
				      the distributive law}                                                                            \\
				       & = (A \cap (B^c \cap B^c)) \cup ((A \cap B^c) \cap C) & \text{by
				      the associative law for $\cap$}                                                                  \\
				       & = (A \cap B^c) \cup ((A \cap B^c) \cap C)            & \text{by Idempotent Law for $\cap$}    \\
				       & = A \cap B^c                                         & \text{by Absorption Law}               \\
				       & = A - B                                              & \text{by definition of set difference} \\
			      \end{align*}
		      \end{proof}
	      \end{theo}

	      \newpage
	\item Simplify the expression, citing a set identity in each step:
	      \[
		      ((A \cap (B \cap C)) \cap (A - B)) \cap (B \cup C^c)
	      \]
	      \begin{theo}[Algebraic Proof]
		      \begin{proof}[Proof by Set Identities] Let $A$, $B$, and $C$ be any sets.
			      \[
				      ((A \cap (B \cap C)) \cap (A - B)) \cap (B \cup C^c)
			      \]
			      \begin{align*}
				           & = ((A \cap (B \cup C)) \cap (A - B)) \cap (B \cup C^c)      & \text{by
				      the set difference law}                                                                                           \\
				           & = ((A \cap B^c) \cap (A \cap (B \cup C))) \cap (B \cup C^c)
				           & \text{by the commutative law for $\cap$}                                                                   \\
				           & = (((A \cap B^c) \cap A) \cap (B \cup C)) \cap (B \cup C^c)
				           & \text{by the associative law for $\cap$}                                                                   \\
				           & = ((A \cap (A \cap B^c)) \cap (B \cup C)) \cup (B \cup C^c)
				           & \text{by the commutative law for $\cap$}                                                                   \\
				           & = (((A \cap A) \cap B^c) \cap (B \cup C)) \cap (B \cup C^c) & \text{by the
				      associative law for $\cap$}                                                                                       \\
				           & = ((A \cap B^c) \cap (B \cup C)) \cap (B \cup
				      C^c) & \text{by the idempotent law for $\cup$}                                                                    \\
				           & = (A \cap B^c) \cap (B \cup C)) \cap (B \cup C^c)           & \text{by
				      the associative law for $\cap$}                                                                                   \\
				           & = (A \cap B^c) \cap (B \cup (C \cap C^c))                   & \text{by the
				      distributive law}                                                                                                 \\
				           & = (A \cap B^c) \cap (B \cup \emptyset)                      & \text{by the
				      complement law for $\cap$}                                                                                        \\
				           & = (A \cap B^c) \cap B                                       & \text{by the identity law for $\cup$}
				      \\
				           & = A \cap (B^c \cap B)                                       & \text{by the commutative law for $\cap$}     \\
				           & = A \cap \emptyset                                          & \text{by the complement law for $\cap$}      \\
				           & = \emptyset                                                 & \text{by the universal bound law for $\cap$} \\
			      \end{align*}
		      \end{proof}
	      \end{theo}
\end{enumerate}
\end{document}
% vim: set ft=tex tw=80 ts=2 sts=2 sw=2 noet spell:
