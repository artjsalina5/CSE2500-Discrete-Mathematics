% copyright arturo salinas-aguayo 2025
\documentclass[12pt]{article}

\usepackage{graphicx}
\usepackage{amsmath}
\usepackage{amssymb}
\usepackage{amsthm}
\usepackage{array}
\usepackage{mdframed}
\usepackage{amsfonts}
\usepackage{fancyhdr}
\usepackage{geometry}
\usepackage{subfigure}
\usepackage{caption}
\usepackage{tikz}
\usepackage{bm}
\usepackage{float}

\geometry{letterpaper, margin=1in}
\graphicspath{ {../images/} }

% Header and Footer
\pagestyle{fancy}
\fancyhf{}
\fancyhead[L]{CSE 2500-01: Homework 5}
\fancyhead[R]{Page \thepage}
\setlength{\headheight}{15pt}

\author{Arturo Salinas-Aguayo}
\title{CSE 2500-01: Homework 5}

\usepackage{etoolbox} % Required for \ifstrempty

\newcounter{theo}[section]
\newenvironment{theo}[1][]{%
  \stepcounter{theo}%
  \ifstrempty{#1}%
    {\mdfsetup{%
      frametitle={%
        \tikz[baseline=(current bounding box.east), outer sep=0pt]
          \node[anchor=east,rectangle,fill=blue!20]
          {\strut Theorem~\thetheo};}}%
    }%
    {\mdfsetup{%
      frametitle={%
        \tikz[baseline=(current bounding box.east), outer sep=0pt]
          \node[anchor=east,rectangle,fill=blue!20]
          {\strut Theorem~\thetheo:~#1};}}%
    }%
  \mdfsetup{
    innertopmargin=10pt,
    linecolor=blue!20,
    linewidth=2pt,
    topline=true,
    frametitleaboveskip=-\ht\strutbox,
  }
  \begin{mdframed}[]\relax%
}{\end{mdframed}}

\renewcommand\qedsymbol{\textbf{QED}}
\newtheorem{example}{Example}
% Example block environment
\newenvironment{examp}
{\vspace{0.5cm}
 \hrule
\vspace{0.5cm}
\begin{example}}
{\hrule
\vspace{0.5cm}
\end{example}}

\begin{document}
\newcommand{\closure}[2][3]{%
	{}\mkern#1mu\overline{\mkern-#1mu#2}}
\newcommand\ncoverline[1]{\mkern1mu\overline{\mkern-1mu#1\mkern-1mu}\mkern1mu}
% Title Page
\begin{titlepage}
	\centering
	\vspace*{3cm}
	\huge\textbf{CSE 2500-01: Homework 5}\\
	\vspace{5cm}
	\Large\textbf{Arturo Salinas-Aguayo}\\
	\normalsize
	Spring 2025\\
	Electrical and Computer Engineering Department\\
	\vfill
	\includegraphics[scale=0.1]{uconnlogo}\\
	College of Engineering, University of Connecticut\\
	\scriptsize{Coded in \LaTeX}
	\vspace*{1cm}
\end{titlepage}

\section*{Problems}
\subsection*{Question 1}
Prove the following properties. You should follow the procedures discussed and shown in the class.

\begin{theo}
	\textit{The sum, product, and difference of any two even integers are even.}
	\begin{proof}
		\underline{Suppose:} $m$ and $n$ are any even integers.\newline
		\underline{By Definition} of even, $m = 2r$ and $n = 2s$ for some integers $r$ and $s$.\newline
		\underline{Then}
		\begin{align*}
			m + n & = 2r + 2s  & \text{(by substitution)} \\
			      & = 2(r + s) & \text{(by algebra)}
		\end{align*}
		\begin{align*}
			m \cdot n & = 2r \cdot 2s  & \text{(by substitution)} \\
			          & = 2(r \cdot s) & \text{(by algebra)}
		\end{align*}
		\begin{align*}
			m - n & = 2r - 2s  & \text{(by substitution)} \\
			      & = 2(r - s) & \text{(by algebra)}
		\end{align*}
		\underline{Let} $t = r + s$, $u = r \cdot s$, and $v = r - s$.\\
		\underline{Note:} $t$, $u$, and $v$ are integers because $t$ is a sum of integers, $u$ is a product of integers, and $v$ is the difference between integers which are all integers.\\
		\underline{Hence:}
		\begin{align*}
			 & m + n     = 2t \\
			 & m \cdot n = 2u \\
			 & m - n     = 2v
		\end{align*}
		where $t$, $u$, and $v$ are some integers.\\
		\underline{It follows by definition} of even that $m + n$, $m \cdot n$, and $m - n$ are even.
	\end{proof}
\end{theo}

\begin{theo}
	\textit{The sum and difference of any two odd integers are even.}
	\begin{proof}
		\underline{Suppose:} $m$ and $n$ are any odd integers.\newline
		\underline{By Definition} of odd, $m = 2r + 1$ and $n = 2s + 1$ for some integers $r$ and $s$.\newline
		\underline{Then}
		\begin{align*}
			m + n & = (2r + 1) + (2s + 1) & \text{(by substitution)} \\
			      & = 2r + 2s + 2                                    \\
			      & = 2(r + s + 1)        & \text{(by algebra)}
		\end{align*}
		\begin{align*}
			m - n & = (2r + 1) - (2s + 1) & \text{(by substitution)} \\
			      & = 2r - 2s                                        \\
			      & = 2(r - s)            & \text{(by algebra)}
		\end{align*}
		\underline{Let} $t = r + s + 1$ and $u = r - s$.\\
		\underline{Note:} $t$ and $u$ are integers because $t$ is a sum of integers, $u$ is a difference of integers which are integers.\\
		\underline{Hence:}
		\begin{align*}
			 & m + n     = 2t \\
			 & m - n     = 2u
		\end{align*}
		where $t$ and $u$ are some integers.\\
		\underline{It follows by definition} of even that $m + n$  and $m - n$ are even.
	\end{proof}
\end{theo}

\begin{theo}
	\textit{The product of any two odd integers is odd.}
	\begin{proof}
		\underline{Suppose:} $m$ and $n$ are any odd integers.\\
		\underline{By Definition} of odd, $m = 2r + 1$ and $n = 2s + 1$, for some integers $r$ and $s$.\\
		\underline{Then}
		\begin{align*}
			m \cdot n & = (2r + 1) \cdot (2s + 1) & \text{(by substitution)} \\
			          & = 2(2rs) + 2s + 2r + 1                               \\
			          & = 2(2rs + s + r) + 1      & \text{(by algebra)}
		\end{align*}
		\underline{Let} $t = 2rs + s + r$\\
		\underline{Note} $t$ is the product and sum of integers, which is an integer.\\
		\underline{Hence}
		$m \cdot n = 2t + 1$ where $t$ is some integer.\\
		\underline{It follows by definition} of odd that $m \cdot n$ is odd.
	\end{proof}
\end{theo}

\begin{theo}
	\textit{The product of any even integer and any odd integer is even.}
	\begin{proof}
		\underline{Suppose:} $m$ is any even integer, and $n$ is any odd integer.\\
		\underline{By definition} of even and odd, $m = 2r$ and $n = 2s + 1$, for some integers $r$ and $s$.\\
		\underline{Then}\\
		\begin{align*}
			m \cdot n & = (2r) \cdot (2s + 1) & \text{(by substitution)} \\
			          & = 2(2rs) + 2r                                    \\
			          & = 2(2rs + r)          & \text{(by algebra)}
		\end{align*}
		\underline{Let} $t = 2rs +r$.\\
		\underline{Note} that $t$ is an integer because it is a sum and product of integers.\\
		\underline{Hence} $m \cdot n = 2t$ where $t$ is some integer.\\
		\underline{It follows by definition} of even that $m \cdot n$ is even.
	\end{proof}
\end{theo}

\begin{theo}
	\textit{The sum of any odd integer and any even integer is odd.}
	\begin{proof}
		\underline{Suppose:} $m$ is any odd integer, and $n$ is any even integer.\\
		\underline{By definition} of even and odd, $m = 2r + 1$ and $n = 2s$, for some integers $r$ and $s$.\\
		\underline{Then}
		\begin{align*}
			m + n & = (2r + 1) + 2s & \text{(by substitution)} \\
			      & = 2r + 2s + 1                              \\
			      & = 2(r + s) + 1  & \text{(by algebra)}
		\end{align*}
		\underline{Let} $t = r + s$ \\
		\underline{Note} that t is an integer since it is the product and sum of integers, which is an integer.\\
		\underline{Hence} $m + n = 2t + 1$, where $t$ is some integer.\\
		\underline{It follows by definition} of odd, that $m + n$ is odd.
	\end{proof}
\end{theo}

\begin{theo}
	\textit{The difference of any odd integer minus any even integer is odd.}
	\begin{proof}
		\underline{Suppose:} $m$ is any odd integer, and $n$ is any even integer.\\
		\underline{By definition} of even and odd, $m = 2r + 1$ and $n = 2s$, for some integers $r$ and $s$.\\
		\underline{Then}
		\begin{align*}
			m - n & = (2r + 1) - 2s & \text{(by substitution)} \\
			      & = 2r - 2s + 1                              \\
			      & = 2(r - s) + 1  & \text{(by algebra)}
		\end{align*}
		\underline{Let} $t = r - s$.\\
		\underline{Note} $t$ is an integer as it is the difference of two integers.\\
		\underline{Hence} $m - s = 2t + 1$, where $t$ is some integer.\\
		\underline{It follows by definition} of odd that $m - n$ is odd.
	\end{proof}
\end{theo}

\begin{theo}
	\textit{The difference of any even integer minus any odd integer is odd.}
	\begin{proof}
		\underline{Suppose:} $m$ is any even integer and $n$ is any odd integer.\\
		\underline{By definition} of even and odd, $m = 2r$ and $n = 2s + 1$, for some integers $r$ and $s$.\\
		\underline{Then}
		\begin{align*}
			m - n & = (2r) - (2s + 1)  & \text{(by substitution)} \\
			      & = 2r - 2s - 1                                 \\
			      & = 2(r - s - 1) + 1 & \text{(by algebra)}
		\end{align*}
		\underline{Let} $t = r - s - 1$\\
		\underline{Note} $t$ is an integer since the difference of integers are integers.\\
		\underline{Hence} $m - n = 2t + 1$, where $t$ is some integer.\\
		\underline{It follows by definition} of odd that $m - n$ is odd.
	\end{proof}
\end{theo}


\end{document}
