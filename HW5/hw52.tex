% Copyright Arturo Salinas-Aguayo 2025
\documentclass[12pt]{article}

\usepackage{graphicx}
\usepackage{amsmath}
\usepackage{amssymb}
\usepackage{amsthm}
\usepackage{array}
\usepackage{mdframed}
\usepackage{amsfonts}
\usepackage{fancyhdr}
\usepackage{geometry}
\usepackage{subfigure}
\usepackage{caption}
\usepackage{tikz}
\usepackage{bm}
\usepackage{float}

\geometry{letterpaper, margin=1in}
\graphicspath{ {../images/} }

% Header and Footer
\pagestyle{fancy}
\fancyhf{}
\fancyhead[L]{CSE 2500-01: Homework 5}
\fancyhead[R]{Page \thepage}
\setlength{\headheight}{15pt}

\author{Arturo Salinas-Aguayo}
\title{CSE 2500-01: Homework 5}

% Custom Theorem Environment
\newcounter{theo}[section]
\newenvironment{theo}[1][]{%
  \stepcounter{theo}%
  \mdfsetup{
    innertopmargin=10pt,
    linecolor=blue!20,
    linewidth=2pt,
    topline=true,
    frametitle={%
      \tikz[baseline=(current bounding box.east).outer sep=0pt]
      \node[anchor=east,rectangle,fill=blue!20]
      {\strut Theorem ~\thetheo#1};
    },
    frametitleaboveskip=0pt,
  }
  \begin{mdframed}[]
}{\end{mdframed}}

% Theorem and Example Definitions
\newtheorem{theorem}{Theorem}
\renewcommand\qedsymbol{\textbf{QED}}
\newtheorem{example}{Example}

\newenvironment{examp}{
  \vspace{0.5cm}
  \hrule
  \vspace{0.5cm}
  \begin{example}
}{
  \hrule
  \vspace{0.5cm}
  \end{example}
}

\begin{document}

% New Commands
\newcommand{\closure}[2][3]{\mkern#1mu\overline{\mkern-#1mu#2}}
\newcommand{\ncoverline}[1]{\mkern1mu\overline{\mkern-1mu#1\mkern-1mu}\mkern1mu}

% Title Page
\begin{titlepage}
	\centering
	\vspace*{3cm}
	\huge\textbf{CSE 2500-01: Homework 5}\\
	\vspace{5cm}
	\Large\textbf{Arturo Salinas-Aguayo}\\
	\normalsize
	Spring 2025\\
	Electrical and Computer Engineering Department\\
	\vfill
	\includegraphics[scale=0.1]{uconnlogo}\\
	College of Engineering, University of Connecticut\\
	\scriptsize{Coded in \LaTeX}
	\vspace*{1cm}
\end{titlepage}

\section*{Problems}
\subsection*{Question 1}
Prove the following properties. You should follow the procedures discussed and shown in the class.

\begin{theorem}
	The sum, product, and difference of any two even integers are even.
\end{theorem}
\begin{proof}
	\underline{Suppose:} $m$ and $n$ are any even integers.\newline
	\underline{By Definition} of even, $m = 2r$ and $n = 2s$ for some integers $r$ and $s$.\newline
	\underline{Then}
	\begin{align*}
		m + n & = 2r + 2s  & \text{(by substitution)} \\
		      & = 2(r + s) & \text{(by algebra)}
	\end{align*}
	\underline{Hence,} the sum is even. Similarly, multiplication and subtraction follow the same pattern.
\end{proof}

\begin{theorem}
	The sum and difference of any two odd integers are even.
\end{theorem}
\begin{proof}
	\underline{Suppose:} $m$ and $n$ are any odd integers.\newline
	\underline{By Definition} of odd, $m = 2r + 1$ and $n = 2s + 1$ for some integers $r$ and $s$.\newline
	\underline{Then}
	\begin{align*}
		m + n & = (2r + 1) + (2s + 1) & \text{(by substitution)} \\
		      & = 2(r + s + 1)        & \text{(by algebra)}
	\end{align*}
	\underline{Thus,} the sum is even. Similar steps follow for subtraction.
\end{proof}

\begin{theorem}
	The product of any two odd integers is odd.
\end{theorem}
\begin{proof}
	\underline{Suppose:} $m = 2r + 1$ and $n = 2s + 1$.\newline
	\begin{align*}
		m \cdot n & = (2r + 1)(2s + 1)   \\
		          & = 4rs + 2r + 2s + 1  \\
		          & = 2(2rs + r + s) + 1
	\end{align*}
	Since $2rs + r + s$ is an integer, $m \cdot n$ is odd.
\end{proof}

\begin{theorem}
	The product of any even integer and any odd integer is even.
\end{theorem}
\begin{proof}
	\underline{Suppose:} $m$ is even and $n$ is odd.\newline
	\begin{align*}
		m \cdot n & = (2r)(2s + 1) \\
		          & = 2(2rs + r)
	\end{align*}
	\underline{Since} $2rs + r$ is an integer, the product is even.
\end{proof}

\end{document}
