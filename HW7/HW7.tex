% copyright arturo salinas-aguayo 2025
\documentclass[12pt]{article}

\usepackage{graphicx}
\usepackage{amsmath}
\usepackage{amssymb}
\usepackage{amsthm}
\usepackage{array}
\usepackage{mdframed}
\usepackage{amsfonts}
\usepackage{fancyhdr}
\usepackage{geometry}
\usepackage{subfigure}
\usepackage{caption}
\usepackage{tikz}
\usepackage{bm}
\usepackage{float}

\geometry{letterpaper, margin=1in}
\graphicspath{ {../images/} }

% Header and Footer
\pagestyle{fancy}
\fancyhf{}
\fancyhead[L]{CSE 2500-01: Homework 7}
\fancyhead[R]{Page \thepage}
\setlength{\headheight}{15pt}

\author{Arturo Salinas-Aguayo}
\title{CSE 2500-01: Homework 7}

\usepackage{etoolbox} % Required for \ifstrempty

\newcounter{theo}[section]
\newenvironment{theo}[1][]{%
  \stepcounter{theo}%
  \ifstrempty{#1}%
    {\mdfsetup{%
      frametitle={%
        \tikz[baseline=(current bounding box.east), outer sep=0pt]
          \node[anchor=east,rectangle,fill=blue!20]
          {\strut Theorem~\thetheo};}}%
    }%
    {\mdfsetup{%
      frametitle={%
        \tikz[baseline=(current bounding box.east), outer sep=0pt]
          \node[anchor=east,rectangle,fill=blue!20]
          {\strut Theorem~\thetheo:~#1};}}%
    }%
  \mdfsetup{
    innertopmargin=10pt,
    linecolor=blue!20,
    linewidth=2pt,
    topline=true,
    frametitleaboveskip=-\ht\strutbox,
  }
  \begin{mdframed}[]\relax%
}{\end{mdframed}}

\renewcommand\qedsymbol{\textbf{QED}}
\newtheorem{example}{Example}
% Example block environment
\newenvironment{examp}
{\vspace{0.5cm}
 \hrule
\vspace{0.5cm}
\begin{example}}
{\hrule
\vspace{0.5cm}
\end{example}}
% Macro for centered equals in final row
\newcommand{\centereq}[2]{#1 \kern-3em = \hphantom{#1} #2}

% Macro for vertical LHS/RHS derivation layout
\newenvironment{lhsrhsproof}[0]{
  \[
  \renewcommand{\arraystretch}{1.5}
  \begin{array}{c@{\quad\quad}c}
  \textbf{LHS} & \textbf{RHS} \\
}{
  \end{array}
  \]
}
\begin{document}
\newcommand{\closure}[2][3]{%
	{}\mkern#1mu\overline{\mkern-#1mu#2}}
\newcommand\ncoverline[1]{\mkern1mu\overline{\mkern-1mu#1\mkern-1mu}\mkern1mu}
% Title Page
\begin{titlepage}
	\centering
	\vspace*{3cm}
	\huge\textbf{CSE 2500-01: Homework 7}\\
	\vspace{5cm}
	\Large\textbf{Arturo Salinas-Aguayo}\\
	\normalsize
	Spring 2025\\
	Electrical and Computer Engineering Department\\
	\vfill
	\includegraphics[scale=0.1]{uconnlogo}\\
	College of Engineering, University of Connecticut\\
	\scriptsize{Coded in \LaTeX}
	\vspace*{1cm}
\end{titlepage}

\section*{Problems}
\begin{theo}[Induction Proof 1]
	Prove the following statement using mathematical induction:
	\[
		\forall n \in \mathbb{Z}, n \geq 1, \quad \frac{1}{1 \cdot 2} + \frac{1}{2 \cdot 3} + \cdots + \frac{1}{n(n+1)} = \frac{n}{n+1}
	\]
	\begin{proof}[Proof by Mathematical Induction.]
		\leavevmode\par
		\noindent Let the property, \( P(n) \) be the equation:\\
		\[
			\frac{1}{1 \cdot 2} + \frac{1}{2 \cdot 3} + \cdots + \frac{1}{n(n+1)} = \frac{n}{n+1}
		\]
		\textbf{Base Case:} Show that \( P(1) \) is true.\\
		\[
			\begin{array}{c@{\quad\quad}c}
				\textbf{LHS}        & \textbf{RHS}                      \\
				\frac{1}{1(1 + 1)}  & \frac{1}{1 + 1}                   \\
				\rotatebox{90}{$=$} & \rotatebox{90}{$=$}               \\
				\multicolumn{2}{c}{\frac{1}{2} \quad=\quad \frac{1}{2}} \\
			\end{array}
		\]
		Therefore, $P(1)$ is true.\\
		\textbf{Inductive Step:} Assume that \( P(k) \) is true for some integer \( k \geq 1 \). That is, assume:
		\[
			\quad \frac{1}{1 \cdot 2} + \frac{1}{2 \cdot 3} + \cdots + \frac{1}{k(k+1)} = \frac{k}{k+1}
		\]
		We need to show that \( P(k + 1) \) is true.\\
		\noindent
		\begin{minipage}[t]{0.48\textwidth}
			\fbox{%
				\begin{minipage}{\textwidth}
					\centering
					\footnotesize
					\textbf{LHS}
					\[
						\begin{array}{c}
							\frac{1}{1 \cdot 2} + \frac{1}{2 \cdot 3} + \cdots + \frac{1}{(k+1)((k+1)+1)} \\
							\rotatebox{90}{$=$}                                                           \\
							\frac{k}{k+1} + \frac{1}{(k+1)((k+1)+1)}                                      \\
							\rotatebox{90}{$=$}                                                           \\
							\frac{k(k+1) + 1}{(k+1)(k+2)}                                                 \\
							\rotatebox{90}{$=$}                                                           \\
							\frac{k^2 + 2k + 1}{(k+1)(k+2)}                                               \\
							\rotatebox{90}{$=$}                                                           \\
							\frac{(k+1)^2}{(k+1)(k+2)}                                                    \\
							\rotatebox{90}{$=$}                                                           \\
							\frac{k+1}{k+2}
						\end{array}
					\]
				\end{minipage}
			}
		\end{minipage}
		\hspace{0.7em}
		\begin{minipage}[t]{0.45\textwidth}
			\fbox{%
				\begin{minipage}{\textwidth}
					\centering
					\footnotesize
					\textbf{RHS}
					\[
						\begin{array}{c}
							\frac{k+1}{(k+1)+1} \\
							\rotatebox{90}{$=$} \\
							\frac{k+1}{k+2}     \\
						\end{array}
					\]
				\end{minipage}
			}
		\end{minipage}\\
		The LHS is equal to the RHS.\\
		Thus, we have shown that if \( P(k) \) is true, then \( P(k + 1) \) is also true.\\
	\end{proof}
\end{theo}

\vspace{1cm}

\begin{theo}[Induction Proof 2]
	Prove the following statement using mathematical induction:
	\[
		\forall n \in \mathbb{Z}, n \geq 0, \quad \sum_{i=1}^{n+1} i \cdot 2^i = n \cdot 2^{n+2} + 2
	\]

	\begin{proof}[Proof by Mathematical Induction.]
		\leavevmode\par
		\noindent Let the property, \( P(n) \) be the equation:
		\[
			\sum_{i=1}^{n+1} i \cdot 2^i = n \cdot 2^{n+2} + 2
		\]
		\textbf{Base Case:} Show that \( P(0) \) is true.\\
		\[
			\begin{array}{c@{\quad\quad}c}
				\textbf{LHS}                 & \textbf{RHS}                     \\
				\sum_{i=1}^{0+1} i \cdot 2^i & 0 \cdot 2^{0+2} + 2              \\
				\rotatebox{90}{$=$}          & \rotatebox{90}{$=$}              \\
				1\cdot2^i                    & 2                                \\
				\rotatebox{90}{$=$}          & \Downarrow                       \\
				2                            & \kern-3em = \quad \hphantom{2} 2 \\
			\end{array}
		\]
		\textbf{Inductive Step:} Assume that \( P(k) \) is true for some integer \( k \geq 0 \). That is, assume:
		\[
			\sum_{i=1}^{k+1} i \cdot 2^i = k \cdot 2^{k+2} + 2
		\]
		Then we need to show that \( P(k + 1) \) is true.\\
		\noindent
		\begin{minipage}[t]{0.48\textwidth}
			\fbox{%
				\begin{minipage}{\textwidth}
					\centering
					\footnotesize
					\textbf{LHS}
					\[
						\begin{array}{c}
							\sum_{i=1}^{(k+1)+1} i \cdot 2^i                   \\
							\rotatebox{90}{$=$}                                \\
							\sum_{i=1}^{k+2} i \cdot 2^i                       \\
							\rotatebox{90}{$=$}                                \\
							\sum_{i=1}^{k+1} i \cdot 2^i + (k+2) \cdot 2^{k+2} \\
							\rotatebox{90}{$=$}                                \\
							k \cdot 2^{k+2} + 2 + (k+2) \cdot 2^{k+2}          \\
							\rotatebox{90}{$=$}                                \\
							2^{k+2}(k + (k+2)) + 2                             \\
							\rotatebox{90}{$=$}                                \\
							2^{k+2}(2k + 2) + 2                                \\
							\rotatebox{90}{$=$}                                \\
							2^{k+3}(k + 1) + 2                                 \\
							\rotatebox{90}{$=$}                                \\
							2^{k+3}(k + 1) + 2
						\end{array}
					\]
				\end{minipage}
			}
		\end{minipage}
		\hspace{0.7em}
		\begin{minipage}[t]{0.48\textwidth}
			\fbox{%
				\begin{minipage}{\textwidth}
					\centering
					\footnotesize
					\textbf{RHS}
					\[
						\begin{array}{c}
							(k + 1) \cdot 2^{(k + 1) + 2} + 2 \\
							\rotatebox{90}{$=$}               \\
							(k + 1) \cdot 2^{k + 3} + 2       \\
							\rotatebox{90}{$=$}               \\
							2^{k + 3}(k + 1) + 2              \\
						\end{array}
					\]
				\end{minipage}
			}
		\end{minipage}\\
		The LHS is equal to the RHS.\\
		Thus, we have shown that if \( P(k) \) is true, then \( P(k + 1) \) is also true.
	\end{proof}
\end{theo}

\vspace{1cm}

\begin{theo}[Induction Proof 3]
	Prove the following inequality using mathematical induction:
	\[
		\forall n \in \mathbb{Z}, n \geq 2, \forall x \in \mathbb{R}, x \geq -1, \quad 1 + nx \leq (1 + x)^n
	\]
	\begin{proof}[Proof by Mathematical Induction.]
		\leavevmode\par
		\noindent Let the property, \( P(n) \) be the inequality:
		\[
			1 + nx \leq (1 + x)^n
		\]
		\textbf{Base Case:} Show that \( P(2) \) is true.\\
		\[
			\begin{array}{c@{\quad\quad}c}
				\textbf{LHS}        & \textbf{RHS}        \\

				1 + 2x              & (1 + x)^2           \\

				\rotatebox{90}{$=$} & \rotatebox{90}{$=$} \\

				1 + 2x              & 1 + 2x + x^2        \\
			\end{array}
		\]
		\[
			1 + 2x \leq 1 + 2x + x^2 \quad \text{is true for } x \geq -1
		\]
		\textbf{Inductive Step:} Assume that \( P(k) \) is true for some integer \( k \geq 2 \). That is, assume:
		\[
			1 + kx \leq (1 + x)^k
		\]
		We must show that $P(k + 1)$ is true.\\

		\noindent
		\begin{minipage}[t]{0.45\textwidth}
			\fbox{%
				\begin{minipage}{\textwidth}
					\centering
					\footnotesize
					\textbf{LHS}
					\[
						\begin{array}{c}
							1 + (k + 1)x        \\
							\rotatebox{90}{$=$} \\
							1 + kx + x          \\
							\rotatebox{90}{$=$} \\
							1 + x(k + 1)
						\end{array}
					\]
				\end{minipage}
			}
		\end{minipage}
		\hspace{.7em}
		\begin{minipage}[t]{0.5\textwidth}
			\fbox{%
				\begin{minipage}{\textwidth}
					\centering
					\footnotesize
					\textbf{RHS}
					\[
						\begin{array}{c}
							(1 + x)^{k + 1}                             \\
							\rotatebox{90}{$=$}                         \\
							(1 + x)(1 + x)^k                            \\
							\Downarrow \text{(by inductive hypothesis)} \\
							(1+x)(1+kx) \leq (1+x)(1+x)^k               \\
							\rotatebox{90}{$=$}                         \\
							(kx^2 + kx + x + 1)  \leq (1+x)(1+x)^k      \\
							\rotatebox{90}{$=$}                         \\
							kx^2+ x(k+1) + 1 \leq (1+x)(1+x)^k          \\
						\end{array}
					\]
				\end{minipage}
			}
		\end{minipage}
		\\
		Since \(  1 + x(k+1) \leq kx^2+ x(k+1) + 1 \), it follows that:
		\[
			1 + x(k+1) \leq kx^2 + x(k+1) + 1 \leq (1+x)(1+x)^k = (1+x)^{k+1}
		\]
		Thus, we have shown that if \( P(k) \) is true, then \( P(k + 1) \) is also true.\\
	\end{proof}
\end{theo}

\newpage

\begin{theo}[Recursively Defined Sequence]
	A sequence \( c_0, c_1, c_2, \dots \) is defined recursively as follows:
	\[
		c_0 = 3, \quad c_k = (c_{k-1})^2 \text{ for all integers } k \geq 1.
	\]
	Prove by induction that:
	\[
		c_n = 3^{2^n} \quad \text{for all integers } n \geq 0.
	\]
	\begin{proof}[Proof by Mathematical Induction.]
		\leavevmode\par
		\noindent	Let the property \( P(n) \) be the equation:
		\[
			c_n = 3^{2^n} \text{ for all integers } n \geq 0.
		\]
		\textbf{Base Case:} Show that \( P(0) \) is true.\\
		\[
			c_0 = 3 \quad \text{and} \quad 3^{2^0} = 3^1 = 3
		\]
		\textbf{Inductive Step:} Assume that \( P(k) \) is true for some integer \( k \geq 0 \). That is, assume:
		\[
			c_k = 3^{2^k}
		\]
		We must show that \( P(k + 1) \) is true.\\
		By Substitution:\\
		\noindent
		\begin{minipage}[t]{0.45\textwidth}
			\fbox{%
				\begin{minipage}{\textwidth}
					\centering
					\textbf{LHS}
					\[
						\begin{array}{c}
							c_{k+1}                                     \\
							\rotatebox{90}{$=$}                         \\
							(c_k)^2                                     \\
							\Downarrow \text{(by inductive hypothesis)} \\
							(3^{2^k})^2                                 \\
							\rotatebox{90}{$=$}                         \\
							3^{2^{k+1}}                                 \\
						\end{array}
					\]
				\end{minipage}
			}
		\end{minipage}
		\hspace{.7em}
		\begin{minipage}[t]{0.5\textwidth}
			\fbox{%
				\begin{minipage}{\textwidth}
					\centering
					\textbf{RHS}
					\[
						\begin{aligned}
							3^{2^{k+1}} \\
						\end{aligned}
					\]
				\end{minipage}
			}
		\end{minipage}\\
		The LHS is equal to the RHS.\\
		Thus, we have shown that if \( P(k) \) is true, then \( P(k + 1) \) is also true.\\
	\end{proof}
\end{theo}
\end{document}
