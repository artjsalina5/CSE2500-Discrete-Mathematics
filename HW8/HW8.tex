% copyright arturo salinas-aguayo 2025
\documentclass[12pt]{article}

\usepackage{graphicx}
\usepackage{amsmath}
\usepackage{amssymb}
\usepackage{amsthm}
\usepackage{array}
\usepackage{mdframed}
\usepackage{amsfonts}
\usepackage{fancyhdr}
\usepackage{geometry}
\usepackage{subfigure}
\usepackage{caption}
\usepackage{tikz}
\usepackage{bm}
\usepackage{float}

\geometry{letterpaper, margin=1in}
\graphicspath{ {../images/} }

% Header and Footer
\pagestyle{fancy}
\fancyhf{}
\fancyhead[L]{CSE 2500-01: Homework 7}
\fancyhead[R]{Page \thepage}
\setlength{\headheight}{15pt}

\author{Arturo Salinas-Aguayo}
\title{CSE 2500-01: Homework 7}

\usepackage{etoolbox} % Required for \ifstrempty

\newcounter{theo}[section]
\newenvironment{theo}[1][]{%
  \stepcounter{theo}%
  \ifstrempty{#1}%
    {\mdfsetup{%
      frametitle={%
        \tikz[baseline=(current bounding box.east), outer sep=0pt]
          \node[anchor=east,rectangle,fill=blue!20]
          {\strut Theorem~\thetheo};}}%
    }%
    {\mdfsetup{%
      frametitle={%
        \tikz[baseline=(current bounding box.east), outer sep=0pt]
          \node[anchor=east,rectangle,fill=blue!20]
          {\strut Theorem~\thetheo:~#1};}}%
    }%
  \mdfsetup{
    innertopmargin=10pt,
    linecolor=blue!20,
    linewidth=2pt,
    topline=true,
    frametitleaboveskip=-\ht\strutbox,
  }
  \begin{mdframed}[]\relax%
}{\end{mdframed}}

\renewcommand\qedsymbol{\textbf{QED}}
\newtheorem{example}{Example}
% Example block environment
\newenvironment{examp}
{\vspace{0.5cm}
 \hrule
\vspace{0.5cm}
\begin{example}}
{\hrule
\vspace{0.5cm}
\end{example}}
% Macro for centered equals in final row
\newcommand{\centereq}[2]{#1 \kern-3em = \hphantom{#1} #2}

% Macro for vertical LHS/RHS derivation layout
\newenvironment{lhsrhsproof}[0]{
  \[
  \renewcommand{\arraystretch}{1.5}
  \begin{array}{c@{\quad\quad}c}
  \textbf{LHS} & \textbf{RHS} \\
}{
  \end{array}
  \]
}
\begin{document}
\newcommand{\closure}[2][3]{%
	{}\mkern#1mu\overline{\mkern-#1mu#2}}
\newcommand\ncoverline[1]{\mkern1mu\overline{\mkern-1mu#1\mkern-1mu}\mkern1mu}
% Title Page
\begin{titlepage}
	\centering
	\vspace*{3cm}
	\huge\textbf{CSE 2500-01: Homework 7}\\
	\vspace{5cm}
	\Large\textbf{Arturo Salinas-Aguayo}\\
	\normalsize
	Spring 2025\\
	Electrical and Computer Engineering Department\\
	\vfill
	\includegraphics[scale=0.1]{uconnlogo}\\
	College of Engineering, University of Connecticut\\
	\scriptsize{Coded in \LaTeX}
	\vspace*{1cm}
\end{titlepage}

\section*{Problems}

A sequence $g_1, g_2, g_3, \dots$ is defined by letting $g_1 = 3$, $g_2 = 5$, and $g_k = 3g_{k-1} - 2g_{k-2}$ for all integers $k \geq 3$.


Prove that $g_n = 2^n + 1$ for all integers $n \geq 1$ using strong mathematical induction.
\begin{theo}[Strong Induction Proof 1]
	\centering\textit{$g_n = 2^n + 1$ for all integers $n \geq 1$ }
	\begin{proof}[Proof by Strong Mathematical Induction]
		\leavevmode\par
		\noindent Let $P(n)$ be the expression:

		\[g_n = 2^n + 1 \quad\text{for all integers}\quad n \geq 1\]

		\noindent\textbf{Base Cases:}\\
		We must show $P(1)$ and $P(2)$ are true.
		\[
			g_1 = 2^1 + 1 = 3 \]
		\[
			g_2 = 2^2 + 1 = 5
		\]
		Both adhere to our initial conditions, so we may proceed.\\
		\noindent\textbf{Inductive Step:}\\
		For an arbitrarily fixed integer $k \geq 2$, suppose $P(k)$ is true, that is: \\
		\[
			g_i = s^i + 1 \quad\text{for all integers}\quad 1 \leq i \leq k.
		\]
		We need to show $P(k+1)$ is true.\\
		That is,
		\[
			g_{k+1} = s^{k+1} + 1
		\]
		From the definition of the sequence,
		\[
			g_{k+1} = 3g_{k} - 2g_{k-1}
		\]
		\begin{align*}
			g_{k+1} & = 3(2^k + 1) - 2(2^{k-1} + 1)    & \text{\textit{(by inductive hypothesis)}} \\
			        & = 3(2^k + 1) - 2\cdot2^{k-1} - 2 &                                           \\
			        & = 3\cdot2^k + 3 - 2^k - 2        &                                           \\
			        & = 2\cdot2^k + 1                  &                                           \\
			        & = 2^{k+1} + 1                    &                                           \\
		\end{align*}
		Since both the basis and inductive steps are true, $P(n)$ is true for all integers $n \geq 1$.
	\end{proof}

\end{theo}

\newpage
Suppose  $d_1, d_2, d_3, \dots$ is a sequence defined by the formula
$d_n = 3^n - 2^n$, for all integers $n \geq 0$.

Show that this sequence satisfies the recurrence relation for $k \geq 2$.

\begin{theo}[Recurrence Relation]
	\textit{For every integer $n \geq 0$, $d_n = 3^n - 2^n$ satisfies $d_k = 5d_{k-1} - 6d_{k-2}$ for $k \geq 2$.}\\
	Then for each integer $k \geq 2$,
	\begin{align*}
		d_k     & = 3^k - 2^k         \\
		d_{k-1} & = 3^{k-1} - 2^{k-1} \\
		d_{k-2} & = 3^{k-2} - 2^{k-2} \\
	\end{align*}
	It follows that for each integer $k \geq 2$,
	\begin{align*}
		5d_{k-1} - 6d_{k-2} & = 5(3^{k-1} - 2^{k-1}) - 6(3^{k-2} - 2^{k-2})                               & \text{\textit{(By Subsititution)}} \\
		                    & = 5\cdot3^{k-1} - 5\cdot2^{k-1} - 2\cdot3\cdot3^{k-2} + 3\cdot2\cdot2^{k-2}                                      \\
		                    & = 5\cdot3^{k-1} - 5\cdot2^{k-1} - 2\cdot3^{k-1} + 3\cdot2^{k-1}                                                  \\
		                    & = 3\cdot3^{k-1} - 2\cdot2^{k-1}                                                                                  \\
		                    & = 3^{k} - 2^{k}                                                                                                  \\
		                    & = d_k                                                                                                            \\
	\end{align*}
	Thus $d_k = 5d_{k-1} - 6d_{k-2}$ for every integer $k \geq 2$.
\end{theo}

\begin{enumerate}
	\item Use iteration to guess an explicit formula for the recurrence relation
	      \(b_k = \frac{b_{k-1}}{1+b_{k-1}}\) for \(k \geq 1\) with \(b_1 = 1\).
	\item Use mathematical induction to show that your guessed formula is correct for $k \geq 0$.
\end{enumerate}
\begin{theo}[Strong Induction Proof 3]

\end{theo}
\end{document}
% vim: set ft=tex tw=80 ts=2 sts=2 sw=2 noet spell:
