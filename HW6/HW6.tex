% copyright arturo salinas-aguayo 2025
\documentclass[12pt]{article}

\usepackage{graphicx}
\usepackage{amsmath}
\usepackage{amssymb}
\usepackage{amsthm}
\usepackage{array}
\usepackage{mdframed}
\usepackage{amsfonts}
\usepackage{fancyhdr}
\usepackage{geometry}
\usepackage{subfigure}
\usepackage{caption}
\usepackage{tikz}
\usepackage{bm}
\usepackage{float}

\geometry{letterpaper, margin=1in}
\graphicspath{ {../images/} }

% Header and Footer
\pagestyle{fancy}
\fancyhf{}
\fancyhead[L]{CSE 2500-01: Homework 6}
\fancyhead[R]{Page \thepage}
\setlength{\headheight}{15pt}

\author{Arturo Salinas-Aguayo}
\title{CSE 2500-01: Homework 6}

\usepackage{etoolbox} % Required for \ifstrempty

\newcounter{theo}[section]
\newenvironment{theo}[1][]{%
  \stepcounter{theo}%
  \ifstrempty{#1}%
    {\mdfsetup{%
      frametitle={%
        \tikz[baseline=(current bounding box.east), outer sep=0pt]
          \node[anchor=east,rectangle,fill=blue!20]
          {\strut Theorem~\thetheo};}}%
    }%
    {\mdfsetup{%
      frametitle={%
        \tikz[baseline=(current bounding box.east), outer sep=0pt]
          \node[anchor=east,rectangle,fill=blue!20]
          {\strut Theorem~\thetheo:~#1};}}%
    }%
  \mdfsetup{
    innertopmargin=10pt,
    linecolor=blue!20,
    linewidth=2pt,
    topline=true,
    frametitleaboveskip=-\ht\strutbox,
  }
  \begin{mdframed}[]\relax%
}{\end{mdframed}}

\renewcommand\qedsymbol{\textbf{QED}}
\newtheorem{example}{Example}
% Example block environment
\newenvironment{examp}
{\vspace{0.5cm}
 \hrule
\vspace{0.5cm}
\begin{example}}
{\hrule
\vspace{0.5cm}
\end{example}}

\begin{document}
\newcommand{\closure}[2][3]{%
	{}\mkern#1mu\overline{\mkern-#1mu#2}}
\newcommand\ncoverline[1]{\mkern1mu\overline{\mkern-1mu#1\mkern-1mu}\mkern1mu}
% Title Page
\begin{titlepage}
	\centering
	\vspace*{3cm}
	\huge\textbf{CSE 2500-01: Homework 6}\\
	\vspace{5cm}
	\Large\textbf{Arturo Salinas-Aguayo}\\
	\normalsize
	Spring 2025\\
	Electrical and Computer Engineering Department\\
	\vfill
	\includegraphics[scale=0.1]{uconnlogo}\\
	College of Engineering, University of Connecticut\\
	\scriptsize{Coded in \LaTeX}
	\vspace*{1cm}
\end{titlepage}

\section*{Problems}
\begin{theo}
	\center\textit{For all integers $k$ with $k \geq 4$, $2k^2 - 5k + 2$ is not prime.}
	\begin{proof}
		Suppose $k$ is any integer such that $k \geq 4$.\\

		\noindent By definition of prime, a number $p \in \mathbb{Z}^+$ is prime if and only if $p > 1$, and for all positive integers $x$ and $y$ such that $xy = p$, either $x = 1$ and $y = p$, or $x = p$ and $y = 1$.\\

		\begin{align*}
			2k^2 - 5k + 2 & = (2k - 1)(k - 2) \quad \text{(by factoring)}
		\end{align*}

		\noindent Since $k \geq 4$
		\begin{align*}
			 & k - 2  \geq 2                           \\
			 & 2k     \geq 8 \Rightarrow 2k - 1 \geq 7
		\end{align*}

		\noindent Let $a = 2k - 1$ and $b = k - 2$. Then $a \cdot b = 2k^2 - 5k + 2$, and both $a, b \in \mathbb{Z}^+$ with $a \geq 7$ and $b \geq 2$.\\
		\noindent Hence, $2k^2 - 5k + 2$ is a product of two positive integers greater than 1.\\

		\noindent It follows by definition that $2k^2 - 5k + 2$ is not prime.
	\end{proof}
\end{theo}
\begin{theo}
	\textit{If $m$ and $n$ are positive integers and $mn$ is a perfect square, then $m$ and $n$ are both perfect squares.}\\

	This is \textbf{False}.\\
	Let $m = n = 2 \rightarrow m \cdot n = 4$\\
	By definition of perfect squares, $4$ is a perfect square since $4 = 2^2$
	but neither $m = 2$ nor $n = 2$ is a perfect square.
\end{theo}
\newpage
\begin{theo}
	\textit{Given two rational numbers $r$ and $s$ where $r < s$, there exists another rational number between $r$ and $s$.}
	\begin{proof}
		Suppose $r$ and $s$ are any two distinct rational numbers.\\
		\noindent By definition of rational numbers, $r = \frac{a}{b}$ and $s = \frac{c}{d}$ for some integers $a$, $b$, $c$, and $d$ with $b \neq 0$ and $d \neq 0$\\
		Then,
		\begin{align*}
			\frac{r + s}{2} & = \frac{\frac{a}{b} + \frac{c}{d}}{2} & \text{(By Substitution.)} \\
			                & = \frac{\frac{ad + bc}{bd}}{2}                                    \\
			                & = \frac{ad + bc}{2bd}                 & \text{(By Algebra.)}
		\end{align*}
		Note that $ad + cd$ and $2bd$ are integers since $a$, $b$, $c$, and $d$ are all integers and the product of integers and sums of integers is an integer. Also, $2bd \neq 0$ by the zero product property.\\

		\noindent It goes to say that by the definition of rational numbers,
		\begin{align}
			\frac{r + s}{2}
		\end{align} is rational.\\

		\noindent Suppose $e$ and $f$ are any real numbers such that $e < f$.\\
		\noindent Dividing into cases and by the properties of inequalities,\\
		\textbf{Case I}\\
		\begin{align*}
			(e + f)         & < 2f \rightarrow \text{Dividing by Two}, \\
			\frac{e + f}{2} & < 2
		\end{align*}
		\textbf{Case II}
		\begin{align*}
			2e & < (e + f) \rightarrow \text{Dividing by Two}, \\
			e  & < \frac{e + f}{2}
		\end{align*}
		Thus, by combination,
		\begin{align}
			e < \frac{e + f}{2} < f
		\end{align}
		Let $y = \frac{r + s}{2}$ where by (1) $y$  is rational, and by (2), $r<y<s$.\\
		Hence there exists another rational number between $r$ and $s$.
	\end{proof}

\end{theo}

\subsection*{Problem 4}
Is $x$ necessarily rational? If so, prove it.
\begin{theo}
	\textit{Let \( a, b, c, d \in \mathbb{Z} \) with \( a \ne c \), and let \( x \in \mathbb{R} \) satisfy the equation
		\[
			\frac{ax + b}{cx + d} = 1.
		\]
		Then \( x \in \mathbb{Q} \); that is, \( x \) is rational.}

	\begin{proof}
		Suppose $a, b, c,$ and $d$ are integers and $a \neq c$.
		Suppose $x \in \mathbb{R}$ and $\frac{ax + b}{cx + d} = 1$\\
		\noindent Then,
		\begin{align*}
			\frac{ax + b}{cx + d} & = 1                   & \text{(Starting Point)}          \\
			ax + b                & = cx + d              & \text{(By Multiplication)}       \\
			x(a - c)              & = (d - b)             & \text{(Separation of Variables)} \\
			x                     & = \frac{d - b}{a - c} & \text{(Algebra)}
		\end{align*}
		\noindent Let $t = \frac{d - b}{a - c}$.
		\noindent Note that t is rational because it is the difference of two integers $d$ and $b$. Also, $a - c \neq 0$ since $a \neq c$.\\
		\noindent It follows by the definition of rational numbers, $t$ is quotient of integers with a non-zero denominator which makes $x$ a rational number since $x = t$.
	\end{proof}

\end{theo}
\subsection*{Problem 5}
\textbf{(5 points)} Use the unique factorization theorem to write the following integers in standard factored form:
\begin{itemize}
	\item[(a)] 1176
	      \begin{align*}
		      1176 = 2^3 \cdot 3 \cdot 7^2
	      \end{align*}
	\item[(b)] 5733
	      \begin{align*}
		      5733 = 3^2 \cdot 7^2 \cdot 13
	      \end{align*}
	\item[(c)] 3675
	      \begin{align*}
		      3675 = 3 \cdot 5^2 \cdot 7^2
	      \end{align*}
\end{itemize}

\subsection*{Problem 6}
\begin{theo}
	\textit{The square of any integer has the form $4k$ or $4k + 1$ for some integer $k$.}
	\begin{proof}
		Suppose $n$ is any integer.\\
		\noindent By the Quotient-Remainder Theorem with a divisor equal to 2,\\
		$n = 2q$ or $n = 2q + 1$ for some integer $q$.\\

		\noindent \textbf{Case I} ($n = 2q$)
		\begin{align*}
			n^2 & = (2q)^2 & \text{(By Substitution)} \\
			    & = 4q^2   & \text{(By Algebra.)}
		\end{align*}
		\noindent Let $k = q^2$. Note that $k$ is an integer because it is a product of integers.\\
		\noindent It follows $n^2 = 4k$ for some integer $k$.\\

		\noindent \textbf{Case II} ($n = 2q + 1$)

		\begin{align*}
			n^2 & = (2q + 1)^2      & \text{(By Substitution)} \\
			    & = (4q^2 + 4q + 1)                            \\
			    & = 4(q^2 + q) + 1  & \text{(By Algebra.)}
		\end{align*}
		\noindent Let $k = q^2 + q$. Note that $k$ is an integer because it is a product of integers.\\
		\noindent It follows $n^2 = 4k + 1$ for some integer $k$.\\

		\noindent Hence, for both cases, there exists an integer $k$ such that $n^2 = 4k$ or $n^2 = 4k + 1$.
	\end{proof}
\end{theo}

\newpage
\subsection*{Problem 7}
``For all integers $a$ and $b$, if $3 \mid (a + b)$, then $3 \mid (a - b)$.''

This is \textbf{False}. To prove this, I prove the inverse to be true.\\
\begin{theo}
	\textit{There exists integers a and b such that $3 \mid (a + b)$ and $3 \nmid (a - b)$.}\\
	\begin{proof}
		\noindent Let $a = 2$ and $b = 1$.\\
		Then,\\
		\begin{align*}
			a + b & = 2 + 1 = 3 & \text{(By Substitution)} \\
		\end{align*}
		Thus, $3 | (a + b)$ because $3 = 3 \cdot 1$.\\
		\noindent Also,
		\begin{align*}
			a - b & = 2 - 1 = 1 & \text{(By Substitution)} \\
		\end{align*}
		But, $3 \nmid 1$ because $\frac{1}{3}$ is not an integer.\\
		Thus it is shown that $3 \nmid (a - b)$.
	\end{proof}
\end{theo}

\subsection*{Problem 8}
\begin{theo}
	\textit{If $n$ is any nonnegative integer whose decimal representation ends in 5, then $5 \mid n$.}
	\begin{proof}
		Suppose $n$ is a nonnegative integer whose decimal representation ends in 5.\\
		Let $n = 10m + 5$ for some integer, $m$.\\
		\begin{align*}
			n & = 10m + 5                            \\
			  & = 5(2m + 1) & \text{(By Factoring.)} \\
		\end{align*}
		Note that the quantity $(2m + 1)$ is an integer since it is the product and sum of integers.\\
		Thus, by definition of divisibility, $n$ is divisible by 5.
	\end{proof}
\end{theo}

\subsection*{Problem 9}
\textbf{(10 points)} Given any integer $n > 3$, could $n$, $n + 2$, and $n + 4$ all be prime? Prove or give a counterexample.\\
They cannot all be prime. To prove this, I will prove the inverse.
\begin{theo}
	\textit{There exists an integer $n > 3$ such that $n$, $n + 2$, or $n + 4$ is not prime.}
	\begin{proof}
		Suppose $n$ is any integer with $n > 3$.\\
		\noindent Let $d = 3$.
		\noindent By the quotient remainder theorem, \\
		$n = 3q$, or $n = 3q + 1$ or $n = 3q + 2$ for some integer $q$.\\
		Note that since $n > 3$, either \\
		\begin{align}[0]
			q & > 1                                        & \text{or} \\
			q & = 1 \text{\quad\ and \quad} n = 4 = 3q + 1 & \text{or} \\
			q & = 1 \text{\quad and\quad} n = 5 = 3q + 1               \\
		\end{align}

	\end{proof}
\end{theo}
\subsection*{Question 1}
Prove the following properties. You should follow the procedures discussed and shown in the class.

\begin{theo}
	\textit{The difference of any even integer minus any odd integer is odd.}
	\begin{proof}
		\underline{Suppose:} $m$ is any even integer and $n$ is any odd integer.\\
		\underline{By definition} of even and odd, $m = 2r$ and $n = 2s + 1$, for some integers $r$ and $s$.\\
		\underline{Then}
		\begin{align*}
			m - n & = (2r) - (2s + 1)  & \text{(by substitution)} \\
			      & = 2r - 2s - 1                                 \\
			      & = 2(r - s - 1) + 1 & \text{(by algebra)}
		\end{align*}
		\underline{Let} $t = r - s - 1$\\
		\underline{Note} $t$ is an integer since the difference of integers are integers.\\
		\underline{Hence} $m - n = 2t + 1$, where $t$ is some integer.\\
		\underline{It follows by definition} of odd that $m - n$ is odd.
	\end{proof}
\end{theo}


\end{document}
